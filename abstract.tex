\begin{center}
\vspace*{18pt}
{\heiti\zihao{-2}\thesistitle}\\[18pt]
\zihao{4}{\kaishu 专业名称}\quad 计算机科学与技术\\[12pt]
{\kaishu 学生} \quad 何为 \quad {\kaishu 指导老师} \quad 张蕾\\[24pt]
\end{center}\par

{\zihao{5}[{\heiti摘要}]随着图像处理和识别技术的发展,字符识别技术被广泛应用到各个领域中。为了实时监控电流表,减少人工成本,需要自动抄表技术。通过摄像头采集电流表图像后,本文通过图像预处理、数字边框提取、数字分割和数字识别这四个过程完成读数识别。图像预处理包括直方图均衡化和图像平滑。数字边框提取则先用阈值法将图像二值化,再通过标记和筛选连通成分找出数字边框的位置,最后用哈夫变换、仿射变换和双线性插值法做倾斜矫正。在字符分割部分,本文介绍了基于边缘和基于动态阈值的数字分割算法。基于边缘的数字分割算法包括边缘检测、轮廓提取和轮廓筛选这几个步骤。基于阈值的分割方法包括动态阈值分割、形态学运算和投影法三个步骤。在数字识别过程中,本文介绍并比较了基于模板匹配和基于特征向量的识别算法。}

\vspace{10pt}

[{\heiti 主题词}]{\kaishu 电流表\quad 图像处理\quad 字符分割\quad 字符识别\quad 二值化\quad}

%%% Local Variables: 
%%% mode: latex
%%% TeX-master: "thesis"
%%% End: 
