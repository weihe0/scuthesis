\begin{center}
\zihao{-2}\thesistitle\\[1ex]
\zihao{4}专业名称\quad 计算机科学与技术\\[1ex]
学生 \quad 何为 \quad 指导老师 \quad 张蕾\\[2ex]
\end{center}\par

{\zihao{5}[{\heiti摘要}]随着图像处理和识别技术的发展,字符识别技术被广泛应用到各个领域中。为了实时监控电流表,减少人工成本,需要自动抄表技术。通过摄像头采集电流表图像后,本文通过图像预处理、数字边框提取、数字分割和数字识别这四个过程完成读数识别。图像预处理包括直方图均衡化和图像平滑。数字边框提取则先用阈值法将图像二值化,再通过标记和筛选连通成分找出数字边框的位置,最后用哈夫变换、仿射变换和双线性插值法做倾斜矫正。在字符分割部分,本文介绍了基于边缘和基于动态阈值的数字分割算法。基于边缘的数字分割算法包括边缘检测、轮廓提取和轮廓筛选这几个步骤。基于阈值的分割方法包括动态阈值分割、形态学运算和投影法三个步骤。在数字识别过程中,本文介绍并比较了基于模板匹配和基于特征向量的识别算法。}

[{\heiti 主题词}]{\kaishu 电流表\quad 图像处理\quad 字符分割\quad 字符识别\quad 二值化\quad}







