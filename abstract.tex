\begin{center}
\vspace*{18pt}
{\heiti\zihao{-2}\thesistitle}\\[18pt]
\zihao{4}{\kaishu 专业名称}\quad 计算机科学与技术\\[12pt]
{\kaishu 学生} \quad 何为 \quad {\kaishu 指导老师} \quad 张蕾\\[24pt]
\end{center}\par

\zihao{5}[{\heiti摘要}]
自动处理和识别图像是一个重要的领域。近几十年来,随着计算机的存储和运算能力的提高,制约图像处理和机器视觉技术应用的障碍逐渐减少,图像处理和机器视觉技术在实际生产生活中得到了广泛的应用。在航空遥感、车牌识别、医学图像、安全防卫、工业制造等领域,都有图像处理和机器视觉技术的应用。例如在制造业中,机器视觉技术常用于检测产品质量。在门禁系统中,可以用图像处理技术识别人脸,确定身份。在图书馆和档案馆,扫描纸质文档后,常用图像识别技术识别文本,以文本而非图片的形式存储文档。

某电站有许多半机械式的电流表。由于电流表数量多,位置分散,且工作在高压环境中,人工抄表费时费力,又有较大的危险。虽然已有电子式电流表,但这些电流表体积庞大,且距离高压线较近,拆卸极为不便且危险。再加上人工抄表的办法难以实现电流的实时监控。综合上述原因,为进行实时监控并减少工作量,在电流表前安装摄像头,从采集的图像中识别电流表读数,是一个可行的办法。

通过摄像头采集电流表图像得到电流表的彩色图像。其中数字只占很小的一部分,而且被周围复杂的区域包围,对读数识别造成干扰。因此分离与数字无关的区域,准确提取数字,直接关系到识别率的高低。针对这种情况,本文将读数识别分成通过图像预处理、数字边框提取、数字分割和数字识别这四个过程。其中前三个步骤均用于消除噪声和排除无关区域。在预处理过程中,首先将摄像头拍摄的彩色图像转换成灰度图像。由于图像含有较多噪声,而且部分图像偏亮或偏暗,所以进行高斯滤波和直方图均衡化处理。在数字边框提取的过程中,首先根据灰度图像中黑色和白色区域面积较大的特点,选用Otsu阈值法得到二值化的图像。二值化的图像包含边框的若干候选区域,用连通成分标记算法从中找出数字边框。由于数字边框有一定的倾斜,用哈夫变换找出倾斜角,然后旋转图像。在字符分割部分,本文使用基于阈值的分割算法。首先用阈值法分割图像,然后做水平和竖直方向的投影,找出每个数字的位置。最后在字符识别的过程中,针对多个电流表的数字形状一致,分割出的数字有部分缺失的情况,提出用模板匹配法识别字符。实验表明,上述方法取得了较好的识别效果。

\vspace{10pt}

[{\heiti 主题词}]{\kaishu 电流表;\quad 图像分割;\quad 字符识别}

%%% Local Variables: 
%%% mode: latex
%%% TeX-master: "thesis"
%%% End: 
