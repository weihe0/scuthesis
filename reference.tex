\begin{thebibliography}{99}\addcontentsline{toc}{chapter}{参考文献}
\bibitem{compvision} Linda Shapiro, George Stockman.计算机视觉[M].赵清杰,钱芳,蔡利栋译.北京:机械工业出版社,2005:76--80.
\bibitem{machvision} Wesley Snyder, Hairong Qi.机器视觉教程[M].林学訚,崔锦实,赵清杰译.北京:机械工业出版社,2005:143--170.
\bibitem{tes} Ray Smith, Hacking Tesseract[EB/OL]. http://tesseract-ocr.repairfaq.org/, 2012-9-15.
\bibitem{imgproc} 刘直芳,王运琼,朱敏.数字图像处理与分析[M].北京:清华大学出版社,2006:616--200
\bibitem{vcpattern} 张宏林. Visual C++数字图像模式识别技术及工程实践[M].北京:人民邮电出版社,2003:145--156.
\bibitem{opencvref} Pisarevsky, OpenCV Reference Manual[EB/OL]. http://docs.opencv.org/open2refman/, 2013--3--15
\bibitem{cookbook} Robert Laganière, OpenCV Cookbook[M]. London:Packet Publishing,2012:7--37
\bibitem{opencvtut} Vardim Pisarevsky, OpenCV Tutorials[EB/OL]. http://docs.opencv.org/tutorials/, 2013--3--15
\bibitem{calculus} 王福楹,王福保,蔡森甫等. 高等数学[M].北京:高等教育出版社,2010,32--33.
\bibitem{vcimg} 苏彦华. Visual C++数字图像识别技术典型案例[M].北京:人民邮电出版社,2003:276--300.
\bibitem{vcimg2} 冯伟兴,梁洪,王臣业. Visual C++数字图像模式识别典型案例详解[M].北京:机械工业出版社,2012:390--432.
\bibitem{alg} Thomas Charles, Charles Leiserson, Ronald Rivest等.算法导论[M].北京:机械工业出版社,2008:305--312.
\bibitem{machmeter} 李辉,基于机器视觉的仪表数字识别研究[D].重庆:重庆大学,2009:29--38.
\bibitem{meter1} 何珣.水表表头数字读数的识别方法研究[D].南京:南京理工大学,2007:14--20.
\bibitem{meter2} 吴梦麟.图像式水表读数识别方法研究[D].南京:南京理工大学,2006:40--44.
\bibitem{position} 曾丽华,李超,熊璋.基于边缘与颜色信息的车牌精确定位算法[J].北京航空航天大学学报,2007,Vol.33:3--11.
\bibitem{otsu} Nobuyuki Otsu. A Threshold Selection Method from Gray-Level Histograms[J]. IEEE Transactions on Systems, Man and Cyber Netics, Vol. SMC-9: 1--5.
\end{thebibliography}
%%% Local Variables: 
%%% mode: latex
%%% TeX-master: "thesis"
%%% End: 
