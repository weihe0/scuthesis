%%% Local Variables: 
%%% mode: latex
%%% TeX-master: t
%%% End: 

\documentclass[adobefonts,a4paper,cs4size,fancyhdr,fntef]{ctexrep}
\usepackage{amsmath,amssymb,algorithm,algorithmic,booktabs,graphicx,paralist,supertabular,subfig,xltxtra,texnames,listings}
\newcommand{\thesistitle}{电流表读数自动识别} %论文标题
\graphicspath{{scu/}{mask/}{process/}} %设置插图目录
\renewcommand{\emph}[1]{\textit{#1}} %fntef选项改变了\emph命令的定义,我再改回来
\renewcommand{\CJKunderlinecolor}{\color{black}} %fntef宏包默认下划线为蓝色,改成黑色
%将版面大小和位置为和Word的默认版面相同。注意,Word的默认版面的有关参数均以英寸(inch)作单位,不是以厘米(cm)作单位。毕业论文模板给出的参数都是错的。连Word都不熟,还好意思做模板。
\usepackage[left=2.5cm,right=2cm,top=2.8cm,bottom=2.8cm]{geometry}
%设置图表标题样式
\usepackage{caption}
\DeclareCaptionFont{hei}{\heiti\zihao{5}}
\captionsetup{font=hei,labelsep=quad}
% 定公式、图、表编号为"3-1"的形式,即分隔符由.变为短杠
\renewcommand\theequation{\arabic{chapter}--\arabic{equation}}
\renewcommand\thefigure{\arabic{chapter}--\arabic{figure}}
\renewcommand\thetable{\arabic{chapter}--\arabic{table}}

%设置章节标题样式。不知道是哪个二货想出的这么难看的样式,不用Word或LaTeX默认的样式会死啊。
\CTEXsetup[name={,},number={\arabic{chapter}},format={\zihao{-3}\centering},beforeskip={-35pt},afterskip={15pt plus 2pt minus 2pt}]{chapter}
\CTEXsetup[format={\zihao{4}\bf\flushleft}]{section}
\CTEXsetup[format={\zihao{-4}\heiti\flushleft}]{subsection}
\CTEXsetup[format={\zihao{-4}\heiti\flushleft}]{subsubsection}
\CTEXsetup[number={\arabic{chapter}}]{appendix} %附录用数字编号
%设置页眉页脚
\pagestyle{fancy}
\fancyhf{}
\fancyhead[L]{\small\bf 四川大学本科毕业论文}
\fancyhead[R]{\small\bf \thesistitle}
\fancyfoot[C]{\thepage} %奇数页的页眉和偶数页的一样,而奇数页的页码和偶数页的页码却是对称摆放。在一般的文献中,页眉页脚要么都对称,要么都一样。再次证明模板的作者是一位完全不懂排版的二货。

%在默认情况下,每一章第一页的页眉不显示,而模板要求显示,所以给这一页加页眉。
\fancypagestyle{plain}
{
  \fancyhf{}
  \fancyhead[L]{\small\bf 四川大学本科毕业论文}
  \fancyhead[R]{\small\bf \thesistitle}
  \fancyfoot[C]{\thepage}
}
\floatname{algorithm}{算法} %将算法标题的标志Algorithm改为“算法”
\renewcommand{\algorithmicrequire}{\textbf{输入}} %把算法的Require改为“输入”
\renewcommand{\algorithmicensure}{\textbf{输出}} %把算法的Ensure改为“输出”
%自定义一组函数。在数学模式下,函数的字体为直立罗马体,变量的字体为斜体。
\DeclareMathOperator{\search}{search}
\DeclareMathOperator{\neighbours}{neighbours}
\DeclareMathOperator{\make}{make}
\DeclareMathOperator{\union}{union}
\DeclareMathOperator{\find}{find}
\DeclareMathOperator{\labelset}{labelset}

%定义GBT7714-2005N标准的参考文献样式
% \usepackage[super,square,numbers,sort&compress]{natbib}
% \bibliographystyle{GBT7714-2005NLang}
\newcommand{\upcite}[1]{\textsuperscript{\cite{#1}}} %将参考文献数字列到右上角
%\usepackage[CJKbookmarks,colorlinks,linkcolor=blue,bookmarks=true,pdfpagemode=UseOutlines,pdfauthor={何为},pdfsubject={数字图像模式识别},pdftitle={\thesistitle},pdfkeywords={图像处理,字符识别}]{hyperref} %hyperref包容易和其他宏包冲突,所以放在最后
\begin{document}
\begin{titlepage}

\newcommand{\titleterm}[2]{\makebox[4em][s]{#1}\CJKunderline{\makebox[14em]{\kaishu #2}}}
\begin{center}
\includegraphics[width=7.28cm,height=1.64cm]{name.png}\\[42.2pt]
\songti\zihao{1} 本科生毕业论文(设计)\\[10.5pt]
\includegraphics[width=3.81cm,height=3.57cm]{brand.png}\\[15.1pt]
\zihao{4}
\newlength{\oldbaselineskip}
\oldbaselineskip=\baselineskip
\baselineskip=2.25\baselineskip
\titleterm{题目}{\thesistitle}\\
\titleterm{学院}{计算机学院}\\
\titleterm{专业}{计算机科学与技术}\\
\titleterm{学生姓名}{何为}\\
\makebox[4em][s]{学号}\CJKunderline{\makebox[7em]{0943041105}}年级\CJKunderline{\makebox[5em]{\kaishu 2009级}}\\
\titleterm{指导教师}{张蕾}\\[26.6pt]
\baselineskip=\oldbaselineskip
\zihao{3}教务处制表\\
\CTEXoptions[today=big]
\zihao{-3}\today
\end{center}
\end{titlepage}

%%% Local Variables: 
%%% mode: latex
%%% TeX-master: "thesis"
%%% End: 
 %封面页
\begin{center}
\vspace*{18pt}
{\heiti\zihao{-2}\thesistitle}\\[18pt]
\zihao{4}{\kaishu 专业名称}\quad 计算机科学与技术\\[12pt]
{\kaishu 学生} \quad 何为 \quad {\kaishu 指导老师} \quad 张蕾\\[24pt]
\end{center}\par

{\zihao{5}[{\heiti摘要}]随着图像处理和识别技术的发展,字符识别技术被广泛应用到各个领域中。为了实时监控电流表,减少人工成本,需要自动抄表技术。通过摄像头采集电流表图像后,本文通过图像预处理、数字边框提取、数字分割和数字识别这四个过程完成读数识别。图像预处理包括直方图均衡化和图像平滑。数字边框提取则先用阈值法将图像二值化,再通过标记和筛选连通成分找出数字边框的位置,最后用哈夫变换、仿射变换和双线性插值法做倾斜矫正。在字符分割部分,本文介绍了基于边缘和基于动态阈值的数字分割算法。基于边缘的数字分割算法包括边缘检测、轮廓提取和轮廓筛选这几个步骤。基于阈值的分割方法包括动态阈值分割、形态学运算和投影法三个步骤。在数字识别过程中,本文介绍并比较了基于模板匹配和基于特征向量的识别算法。}

\vspace{10pt}

[{\heiti 主题词}]{\kaishu 电流表\quad 图像处理\quad 字符分割\quad 字符识别\quad 二值化\quad}

%%% Local Variables: 
%%% mode: latex
%%% TeX-master: "thesis"
%%% End: 
 %中文摘要
\begin{center}
\vspace*{12pt}
{\bf\zihao{-2}The Algorithm and Implementation of \\[10pt]Ammeter Digit Recognition}\\[18pt]
\zihao{4}Computer Science and Technology\\[12pt]
{\bf Student} \quad Wei HE \quad {\bf Advisor} \quad Lei ZHANG\\[24pt]
\end{center}\par

{\zihao{5}[{\bf Abstract}]Image is an important form of information. Automatic image processing and recognition is an important field. In recent decades, as the storages volumn and computation speed of computers have been considerably improved, the obstacles that once prevented the application of image processing and computer vision no longer exist. Because of thia, Image processing and computer vision has been widely employed in plate number recognition, medical image processing, security and manufacturing. For instance, in manufacting, computer vision is often used to inspect the defectes of products. In door security, image processing is used to recognise the identity of guests. In libraries and archieves, printed essays are scanned and their texts are automatically recognised.

There are many semi-mechanical ammeters in a power station. Due to the face that these ammters are scattered and located in high voltage environment, it is arduous and adventurous for workers to read the meters. These semi-machanical ammeters cannot be replaced by electronic ammters in short period, as they are considerably large and installed in cabinet tightly. Because of this, in order to reduce the workload and achieve real-time monitoring, it's essential to recognise the digits in the images of them.

The cameras in front of them capture color image. The digits in the images takes only a small part, which is surrounded by several areas, causing interference in recognition. Therefore, it is a key point to seperate digits and other parts. Considering these factes, the thesis proposed that digit recognition be divided into four stages consequently: preprocessing, frame positioning, digit seperation and digit recognition. During preprocessing stage, the color image is converted to gray-scale image at first. After that, we perform gaussion blur and histogram equalization. During the stage of frame positioning, consider the fact that black and white areas contribute to the most part of the image, we can use Otsu algorithm to acquire binary image. Then we use connected components labeling algorithms to find out the frame. Since the frame is slanted, we invoke Hough transform to determine the slope and calibrate the frame. In the segmentation stage, the thesis proposed a threshold based segmenting algorithm. After that, we perform projections in both horizontal and vertical axis to locate the images. In the last stage, as the digits in ammters are consistent, the thesis proposed an template matching algorithm to recognise the digits. The experiments have proved the effect of the algorithms mentioned above.
}

\vspace{10pt}

[{\bf Key Words}]{Ammeter, Image Segmentation, Charactor Recognition}

%%% Local Variables: 
%%% mode: latex
%%% TeX-master: "thesis"
%%% End: 

 %英文摘要
\zihao{-4}
\tableofcontents

\chapter{绪论}

\section{课题的背景}

各种各样的仪表用于测量系统,其中大多数仪表是机械式和半机械式的。这些仪表不能以数字化的形式存储读数,也没有数据接口传输读数。这些仪表的读数需要人工抄录。在有些场合下,人工抄录是可行的。但在另一些场合下,人工抄录不能满足要求。在需要实时监控的场合,用人工抄录的办法不能及时使响应读数的异常变化,也很难做到全天24小时监控。在化工、钢铁、冶金和核辐射等危险场合,人工读数威胁到工作人员的安全,是不现实的做法。

随着图像处理、模式识别和机器视觉等技术的发展,许多图像识别任务都可以使用计算机自动完成。例如在制造业中,常常利用机器视觉方法自动地检测产品的质量,从而极大地提高生产的自动化程度;在海关、自动柜员机、门禁系统中,通过扫描虹膜或人眼,计算机系统能识别身份;在银行业,机器视觉方法用于自动地识别印章真伪;档案管理系统也可以用机器视觉技术将纸质文档数字化,以便于存储和检索。仪表读数识别也可以利用机器视觉技术完成。通过自动识别仪表读数,可以方便地进行集中管理,及时地对异常情况进行处理。



\section{课题的意义}

本文要识别的电流表来自于发电厂。这些电流表都是半机械式的。由于电流表在高压环境中,人工抄表有较大危险。另一方面,这些电流表体积庞大,且安装在整套的仪表柜中,很难取出,这些半机械式的电流表暂时无法替换成电子式电流表。为了实现实时监控,可以在电流表前安装摄像头,实时采集电流表图像,然后用计算机系统识别。这种方式能减少工作量,实现无人值守的电流监控。随着电子技术和计算机技术的发展,各种图像采集系统和机器识别系统的性能越來越高,价格也具有很强的竞争力。因此,用仪表读数的自动识别代替人工抄表,能降低人力成本,也可以减少高电压的危险环境对工作人员的伤害。

\section{相关工作}

仪表读数识别涉及数字图像处理技术。随着计算机的计算速度和存储能力的迅速发展,图像处理所需的极高的计算和存储要求得到满足。限制图像处理技术应用的因素大多被消除。 数字图像处理技术得到广泛应用。

仪表读数识别和光学字符识别有着很紧密的联系。早在1929年,德国就出现了第一份有关光学字符识别的专利。到五十年代计算机出现时,IBM公司便开始光学字符识别的研究,当时用于扫描美国政府浩如烟海的纸质档案,以文本的形式储存在计算机中。从八十年代开始,CCD和CMOS等图像采集器件和大规模集成电路的出现,使数字图像的采集、存储和处理技术跃上了一个新的台阶,光学字符识别技术也随之得到极大的发展。光学字符识别技术用于档案扫描、信件分拣、车牌识别等方面。特别值得一提的是由惠普研究院研制的tesseract-ocr软件。它能读入多种格式的图像,并能识别超过60种语言的文本图像。在光学字符识别精度的竞赛中,它也名列前茅。

我国在数字识别方面的研究从八十年代开始,起步比国外晚。几十年来,科研人员付出了艰辛的劳动,取得了丰硕的成果。光学字符识别技术取得了长足的进步。如今,对于常见的光学字符应用,已经有了成熟的解决方案。在交通领域,光学字符技术广泛地用于车牌识别。在印刷体字符识别领域,由清华大学研发的TH-OCR获得了业界广泛的认可。目前,光学字符识别理论非常成熟,对于较清晰的字符,识别率已接近100\%,这些为仪表读数的计算机识别奠定了坚实的基础。

%%% Local Variables: 
%%% mode: latex
%%% TeX-master: "thesis"
%%% End: 


\chapter{图像处理基本理论}

电流表识别的各个环节都设计图像处理的基本理论。

\section{图像增强}

实际图像不是完美的。由于噪声和光照等原因,图像的质量达不到自动识别的要求,所以需要进行图像增强。图像增强可以突出待识别的目标的特征,抑制噪声和目标外的部分,从而使增强的图像更适合计算机识别。常见的图像增强方法有\emph{灰度变换}和\emph{图像平滑}。

\subsection{灰度变换}

灰度变换就是将图像每个像素的灰度变换成另一种灰度。灰度变换后每个像素的灰度仅仅由变换前的灰度决定。如果输入图像为$F(i,j)$,输出图像为$G(i,j)$,则灰度变换可以表示为:
\begin{equation}
  G(i,j)=T(F(i,j))
\end{equation}

变换函数$T(x)$可以由用户指定,也可以根据直方图确定。\emph{直方图均衡}就是一种根据直方图确定变换函数的方法。直方图均衡主要用于增加图像的全局对比度,改善光照条件不佳的图像。例如对背景或前景过亮或过暗的图像进行光照补偿,增强曝光过度或曝光不足的图像的细节等。直方图均衡化使输出图像包括所有可能的灰度级,并在每个灰度级上有大致相等的像素个数。要达到上述要求,首先求出直方图。设$n$表示图像的像素数,$n_i$表示灰度$i$出现的次数。灰度$i$有$0,1,\cdots,L-1$共$L$个灰度级。则直方图均衡化的灰度变换函数$T(x)$可以用如下方式求出。直方图$p(i)$表示灰度$i$出现的概率,用公式表示为:
\begin{equation}
  \label{eq:hist}
  p(i)=\frac{n_i}{n},i=0,1,\cdots,L-1
\end{equation}
再从直方图求出累计概率函数$c(i)$,定义如下:
begin{equation}
  \label{eq:acc}
  c(i)=\sum_{j=0}^i p(j)
\end{equation}
最后将$c(i)$映射到$L$个灰度级:
\begin{equation}
  \label{eq:map}
  m(c(i))=\left[\frac{c(i)-\min(c(i))}{\max(c(i))-\min(c(i))}\cdot L\right]
\end{equation}
此时得到灰度变换函数为$T(x)=m(c(x))$。

\subsection{图像平滑}

原图像含有较多的噪声。噪声一般表现为孤立的点和线。用图像平滑的方法,使图像模糊化。去除孤立的点线。图像平滑对像素的邻域进行运算。这种运算可以用模板表示。\emph{模板}是表示领域内像素位置对应的值的方法。


将每个像素的灰度用其邻域的算术平均值代替,可以减少图像的噪声,实现对图像的平滑处理。这种方法称为\emph{盒形滤波}。设$F(i,j)$表示输入图像,$G(i,j)$表示平滑的图像,$S$表示邻域的像素坐标集合,$M$表示邻域的像素数目,则盒形滤波可以用公式表示为:



\section{边缘检测}

\subsection{一维信号边缘检测}

\subsection{二维图像边缘检测算子}

\subsubsection{简单算子}

\subsubsection{Canny算子}


\section{图像分割}

\subsection{阈值分割}

\subsection{轮廓分割}

\section{二值图像分析}

\subsection{连通成分标记}

\subsection{形态学运算}
%%% Local Variables: 
%%% mode: latex
%%% TeX-master: "thesis"
%%% End: 


\chapter{机器视觉库OpenCV介绍}

\section{图像滤波}

\section{几何变换}

\section{结构分析}

\section{特征算子}


%%% Local Variables: 
%%% mode: latex
%%% TeX-master: "thesis"
%%% End: 



\chapter{算法实现}

\section{图像预处理}

受光照条件和噪声等因素的影响,图像的质量不佳,不能直接用于数字识别。需要进行预处理,改善图像质量,从而更容易识别数字。本文采用的预处理方法先后有\emph{灰度化},\emph{高斯滤波}和\emph{直方图均衡}。

\subsection{灰度化}



摄像头采集的图像是彩色图像,如图\ref{fig:src}所示。由于待识别的数字在右上角,所以只需截取右上部分做后续处理,如图\ref{fig:rgb}所示。首先将彩色图像转换成灰度图像。这样做的原因是:
\begin{asparaenum}[(1)]
\item 灰度图像只有一个分量,比三分量的彩色图像更易处理;
\item 本文识别的数字边框为黑底白字,边框内绝大部分像素都是黑白颜色,灰度化后颜色基本没有变化。
\end{asparaenum}

本文使用加权平均法求出灰度图像,公式为$I=0.114R+0.587G+0.299B$。灰度图像如\ref{fig:gray}所示。
\begin{figure}[h]
  \centering
  \includegraphics[scale=0.5]{src.png}
  \caption{原始图像}
  \label{fig:src}
\end{figure}
\begin{figure}[h]
  \centering
  \subfloat[彩色图像]{\label{fig:rgb}\includegraphics[scale=0.5]{rgb.png}}\hspace{1in}
  \subfloat[灰度图像]{\label{fig:gray}\includegraphics[scale=0.5]{gray.png}}
  \caption{灰度化}
\end{figure}

\subsection{高斯滤波}


灰度图像含有较多噪声,不利于提取目标,需要用图像平滑的方法减少噪声。观察发现,待处理图像的噪声以盐椒噪声为主,因此选用高斯滤波进行平滑滤波。在调用OpenCV的高斯滤波函数时,要指定高斯滤波模板的长宽,$\sigma$值和边界扩充类型。一般情况下,只需使用长宽相等的高斯滤波模板。模板尺寸越大,平滑的效果越好,但图像的边缘也越模糊。综合以上情况,将长和宽均设置为3。$\sigma$值的设定也有一定的要求。高斯函数在半径为$2\sigma$范围内的积分为0.95,故绝大部分权值集中在这个领域内。为了最大限度地发挥高斯滤波的效果,需要保证模板能覆盖这个邻域,即$m\sigma$。边界扩充类型指将高斯滤波模板中心与图像边界重合,模板覆盖的部分像素超出图像边界时,这部分像素值的计算方式。一般的做法是将边界内的像素值复制到边界外。高斯滤波的效果如\ref{fig:gauss}所示。
\begin{figure}[h]
  \centering
  \includegraphics[scale=0.5]{gaussian.png}
  \caption{高斯滤波}
  \label{fig:gauss}
\end{figure}

\subsection{直方图均衡}


由于不同时间内光照条件不同,不同时间采集的图像的整体亮度不同。例如日光微弱时图像整体偏暗,日光强烈时图像整体偏亮。偏暗和偏亮的图像对比度不高,不利于识别。因此,采用直方图均衡化的方法增强图像的动态范围,从而达到增强图像整体对比度的效果。实验发现,在进行直方图均衡化前,需要进行图像平滑。这是因为直方图均衡对图像的所有像素的灰度不加选择地加以扩充,因此噪声的灰度也被扩充,使原本灰度相近的区域灰度差距加大,对图像分割造成不利影响。

比较均衡化前的直方图\ref{fig:histsrc}和均衡化后的直方图\ref{fig:histequal}分别是均衡化前和均衡化后的直方图,可以发现均衡化后的直方图比均衡化前的直方图具有更宽的动态范围,每个灰度级上有大致相等的像素数。
\begin{figure}[h]
  \centering
  \subfloat[均衡化前的直方图]{\label{fig:histsrc}\includegraphics[scale=0.5]{histsrc.png}}
  \subfloat[均衡化后的直方图]{\label{fig:histequal}\includegraphics[scale=0.5]{histequal.png}}
  \caption{均衡化前后的直方图}
\end{figure}
直方图均衡化后的图像如图\ref{fig:equal}所示。经直方图均衡化后,不同时间的图像灰度分布大致相同,对于图像分割和识别是十分有益的。
\begin{figure}[h]
  \centering
  \includegraphics[scale=0.5]{equal.png}
  \caption{直方图均衡化}
  \label{fig:equal}
\end{figure}



\section{数字边框定位}

数字边框定位是电表读数识别的一个关键环节,也是较难的环节。数字边框内的数字是待识别的目标,数字边框外的图像则为无用信息。直接从原图中提取数字,难度大,准确率低。因此,应首先找出数字边框,去除无用的部分。观察图\ref{fig:equal}发现,数字边框周围有一些干扰区域,如右侧的电表边框和下侧的指针边框颜色灰度和数字边框的背景类似。这些干扰区域增加了分割的难度。

\subsection{二值化}

数字边框定位的第一步是二值化,即将原图分割为目标区域和背景区域,将目标区域的灰度设为255,背景区域的灰度设为0。观察图像发现,图像的各个区域内灰度差别较小,区域间灰度差别较大,适合用阈值法分割。阈值分割的关键是找到阈值。由于本文的灰度图像由大片的黑色和白色区域组成,可以使用Otsu算法分割阈值,

由式\eqref{eq:otsu}可得,Otsu算法需要对每个可能的灰度$t$计算$q_1(t),q_2(t),\mu_1(t),\mu_2(t)$四项。如果直接根据式\eqref{eq:musig}计算这四项,计算过程复杂而低效。为了简化计算,可以将式\eqref{eq:musig}改写成递推公式:
\begin{equation}
  \label{eq:recursive}
  \begin{aligned}
    q_t(t+1)=q_1(t)+P(t+1)\quad q_2(t+1)=1-q_1(t) \\
    \mu_1(t+1)=\frac{\mu_1(t)q_1(t)+(t+1)q_1(t+1)}{q_1(t+1)} \\
    \mu_2(t+1)=\frac{\mu-q_1(t+1)\mu_1(t+1)}{q_2(t+1)}
  \end{aligned}
\end{equation}
递推公式\eqref{eq:recursive}含有总体均值$\mu$,需要在遍历$t$计算这四项之前计算。设定这四项的初值$q_1(0)=0,q_2(0)=1,\mu_1(t)=0,\mu_2(t)=\mu$,从$t=0$开始计算这四项,每增加一次$t$值,就更新这四项的值。如果小于某个$t$值的区间均值$q_1(t)$极小,则其方差$\mu_1(t)$极大,这个区间的像素极少,这样的$t$值必然不能作为分割点。同理,如果$t$值过大,导致大于$t$的区域的像素极少,同样不能作为阈值。因此计算$q_1(t)$和$q_2(t)$后,如果其中一项小于一个很小的门限,则这时的$t$值不能作为阈值,应直接进入下一轮循环。确保$t$值有效后,再用式\eqref{eq:sigb}更新组间方差$\sigma_B^2$,如果大于目前最大的$\sigma_B^2$值,则将阈值$t^{*}$更新为$t$。遍历$t$的所有取值后,即找出阈值$t^{*}$。分割结果是二值图像,如图\ref{fig:otsu}所示。

\begin{figure}[h]
  \centering
  \includegraphics[scale=0.5]{otsu.png}
  \caption{Otsu算法}
  \label{fig:otsu}
\end{figure}

\subsection{连通成分筛选}

观察图\ref{fig:otsu}可得,二值化的图像黑色部分不仅包含数字边框,还包含其他几个区域,如电流表指针盘的边框和右侧的边框。这些区域需要去除,只保留数字边框。


由于这几个区域都是黑色区域,为了区分这几个区域,可以用连通成分标记算法给不同的区域赋予不同的标记。一般情况下,区域内可能有空洞或裂痕,连通成分标记可能失效。但这里的图像二值化的效果较好,区域内部不存在裂痕,适合用连通成分标记算法处理。本文选用\ref{sec:comp}节介绍的逐行扫描算法进行连通成分标记。由\ref{sec:comp}节可知,在第一轮扫描时,需要记录临时标记的等价关系。在第二轮扫描中,又需要将各点的临时标记替换为等价的标号。由于在第一轮扫描中出现等价的临时标记的次数较多,第二轮扫描对所有的像素点都要找出等价的标号,因此需要用一个能快速实现记录等价关系并快速查找等价标号的数据结构。\emph{并查集}能满足上述要求。

并查集是一种存储不相交的集合的数据结构。它有三个基本操作$\make,find$和$\union$。其中$\make(x)$创建一个仅包含元素$x$的集合,$\find(x)$查找元素$x$所在的集合,$\union(x,y)$合并元素$x$和$y$所在的集合。并查集一般用树的集合表示。树中的每个结点表示集合的一个元素。每个结点只指向其结节点,根节点代表整个集合。实现$\find(x)$操作的一种简单的方法是从结点$x$开始,找到其父节点,再找到父结点的父结点……直到找到根结点,将根结点作为集合的代表。实现$\union(x,y)$操作的一种简单的方法是用$\find(x)$和$\find(y)$操作找到$x$和$y$的根结点,然后将$x$的根结点指向$y$的根结点,反之也行。

用两种策略可以优化$\find(x)$和$\union(x,y)$操作。第一种策略是\emph{按秩合并},对任意元素$x$,它的\emph{秩}定义为从该结点到其后代结点的最长路径的边数的上界,记作$rank[x]$。当用$\make(x)$创建仅包含元素$x$的集合时,$rand[x]=0$。所有$\find(x)$操作不改变$rank[x]$。进行$\union(x,y)$操作时,比较结点$x$和结点$y$的祖先的秩,如果不相等,将秩较低的根结点指向秩较高的根结点,两者的秩不增加;如果相等,则任选一个根结点指向另一个根结点,并增加新的根结点的秩。另一种策略是\emph{路径压缩}。它用于$\find(x)$操作,将其访问的所有的结点都直接指向根结点。

用并查集使逐行扫描算法更加高效。算法第一次扫描,找到种子像素时,设定一个标号,并在并查集中创建一个只包含该标号的集合。在将标号传播到右下方的邻接点的过程中,如果发现两个不同的标号传播到同一个像素,只传播较小的标号,并将两个标号所在的集合合并。第一次扫描结束后,所有标号所在的集合均已确定。在第二次扫描时,将像素的标号改成该标号所在集合的代表标号。

连通成分标记算法的结果如图\ref{eq:candidate}所示,不同区域已经用不同的灰度区分。
\begin{figure}[h]
  \centering
  \includegraphics[scale=0.5]{candidate.png}
  \caption{连通成分}
  \label{fig:candidate}
\end{figure}


标记完各个连通成分后,下一步是从中找出数字边框。对这些连通成分进行分析。首先对连通成分进行初步分析,得到每个连通成分的基本特征,如像素数$n$,连通成分内像素在水平和竖直方向的最小和最大下标$\min(i),\max(i),\min(j),\max(j)$。在这些基本特征的基础上可以得到另一些特征。例如包围连通成分的最小矩形就是从第$\min(i)$行到第$\max(i)$行,第$\min(j)$列到第$\max(j)$列围成的区域。这个矩形的长$L=\max(j)-\min(j)$,宽$W=\max(i)-\min(i)$,长宽比$r=L/W$,面积$S=L\times R$。再定义连通成分的区域密度为$\rho=n/S$。实验发现,数字边框对应的连通成分的各个特征具有如下范围:
\begin{equation}
  \label{eq:range}
  \begin{cases}
    8000  <  p  <  10000 \\
1.9  <  r  <  2.9 \\
0.6  <  \rho  < 0.9 
  \end{cases}
\end{equation}
因此,对每个连通成分计算这些特征值,发现落在式\eqref{eq:range}所示范围的,可以认定为数字边框。找出的数字边框如图\ref{eq:frame}所示。
\begin{figure}[h]
  \centering
  \includegraphics[scale=0.5]{frame.png}
  \caption{数字边框}
  \label{fig:frame}
\end{figure}

\section{数字分割}

目前没有一种通用的分割方法,适用于各个领域。应该根据实际情况和领域知识,灵活地选定一种方案。

\section{数字识别}



%%% Local Variables: 
%%% mode: latex
%%% TeX-master: "thesis"
%%% End: 

%%%
%\CTEXsetup[name={,},number={\arabic{chapter}},format={\zihao{-3}\centering}]{chapter} %正文后面的章标题居然又要改成居中排版。正文用一种样式,文后用另一种,瓜得很。


\chapter{总结}

由于电流表图像中,与读数无关的部分太多,图像中也含有较多噪声,将读数识别分成图像预处理、数字边框提取、数字分割和数字识别这四个部分。

在预处理部分,为增强图像质量,减少噪声,针对电流表图像的特点,使用了几种常见的图像处理方法。摄像头拍摄的图像是彩色图像,处理比较困难,因此先转换成灰度图像。由于电流表图像采集的时间不同,图像的光照条件也不相同,为了减少光照条件变化带来的影响,使各个时间的图像的灰度大致相同,我们采用直方图均衡化处理图像。然后观察图像噪声的特点,发现主要是孤立的点线噪声。根据噪声的这个特点,本文提出进行高斯滤波以见效噪声。

在数字边框提取部分,首先根据仪表可以分成大片的白色和黑色区域的情况,采用Otsu阈值法,找出黑色区域。然后找出这些像素形成的若干个连通成分。为了减少计算开销,本文逐行扫描算法。然后用连通成分的长宽比、像素密度、像素数目和面积等指标筛选连通成分,找出数字边框对应的连通成分。由于数字边框由一定的倾斜角度,先做水平方向的边缘检测,再利用改进的哈夫变换找出上下两条横线的倾斜角度,然后根据倾斜角度利用仿射变换将数字边框图像旋转回原位置。

在数字分割部分,根据背景不均匀的特点,本文分使用了基于动态阈值的数字分割算法。首先用动态阈值法粗略地找出数字对应的像素,再用形态学的方法填补数字的空洞和缝隙,消除错分为数字像素的背景像素。然后做水平和竖直方向上的投影,通过寻找波峰和波谷,找出数字的位置。

在数字识别部分,本文针对仪表数字大小形状一致和数字分割中易出现断裂和缺口的情况,采用模板匹配法识别数字。在进行模板匹配哦前,需要对待识别图像进行归一化,将数字的尺寸、大小和笔画粗细统一。其中在笔画粗细归一化使用Hilditch算法。最后在匹配模板时,使用Hausdorff距离度量待识别数字和模板的差距,取差距最小的模板对应的数字最为识别结果。

%%% Local Variables: 
%%% mode: latex
%%% TeX-master: "thesis"
%%% End: 
 
%\chapter*{作者在读研期间科研成果介绍}
\begin{thebibliography}{99}\addcontentsline{toc}{chapter}{参考文献}
\bibitem[aaa]{sssssss}
\end{thebibliography}
%%% Local Variables: 
%%% mode: latex
%%% TeX-master: "thesis"
%%% End: 


\chapter*{声\qquad明}


本人声明所呈交的学位论文是本人在导师指导下进行的研究工作及取得的研究成果。据我所知,除了文中特别加以标注和致谢的地方外,论文中不包含其他人已经发表或撰写过的研究成果,也不包含为获得四川大学或其他教育机构的学位或证书而使用过的材料。与我一同工作的同志对本研究所做的任何贡献均已在论文中作了明确的说明并表示谢意


\chapter*{致\qquad谢}\addcontentsline{toc}{chapter}{致谢}

首先感谢Linda Shapiro和George Stockman。他们合著的《计算机视觉》深入浅出地讲述了机器视觉的基本理论和技术。我就是从这本书学到图像处理和计算机视觉的基础知识。另外,Wesley Snyder和Hairong Qi合著的《机器视觉教程》内容浅显,让我在短时间内就掌握了机器视觉的基本算法的要领。刘直芳、王运琼和朱敏老师合编的《数字图像处理与分析》包含了丰富的公式。本文的部分公式就来自这本书。张宏林编著的《Visual C++数字图像模式识别技术与工程实践》和苏彦华编著的《Visual C++数字图像识别技术典型案例》包含了大量的源代码。通过阅读源代码,我实现了自己的毕业设计。在此对他们提出感谢。

毕业论文和设计里大量使用了OpenCV的函数。感谢OpenCV的开发者Vadim Pisarevsky等人。OpenCV提供了许多常见的图像处理和机器视觉算法,大大降低了我开发图像识别程序的难度。为学习OpenCV,我参考了《Mastering OpenCV with Practical Computer Vision Projects》和《OpenCV Cookbook》两本书。为此,向下面这几位作者致谢:Daniel Lélis Baggio,Shervin Emami,David Millán Escrivá,Khvedchenia Ievgen,Naureen Mahmood,Jason Saragih,Roy Shilkrot,Robert Laganière。

本文是用\XeLaTeX~排版的。\XeLaTeX~由基础排版引擎\XeTeX~和高层格式包\LaTeX~组成,而\XeTeX~排版引擎源于\TeX~。所以这里感谢\XeTeX~的开发者Jonathan Kew,\LaTeX~的开发者Leslie Lamport和\TeX~的开发者Donald Knuth。\XeLaTeX~本身的功能有限,仅仅用它是无法完成论文排版的。因此我使用了大量的\LaTeX的扩展包,其中最重要的是\LaTeX中文排版扩展包ctex,所以这里特别感谢开发和维护ctex的中文\TeX~学会。在使用\LaTeX~的过程中,《\LaTeXe~完全学习手册》给了我极大的帮助。这本书详细地解释了常见的\LaTeX~命令。每当我用\LaTeX排版遇到问题时,在这本书里总能找到解决方案,所以这里特别感谢这本书的作者胡伟。本文的插图是用MetaPost制作的。感谢MetaPost的作者John D. Hobby和Taco Hoekwater。没有他们,用\XeLaTeX排版出精美的毕业论文,是难以想象的。

感谢指导毕业论文的张蕾老师和郭泉师兄。在和郭泉师兄的交流中,我逐渐明白了图像识别系统的原理和常见问题。另外感谢卢晓春老师,她提出了许多有关图像识别的宝贵建议,使我找到了合适的算法。最后感谢父母对我的关怀和支持。
%%% Local Variables: 
%%% mode: latex
%%% TeX-master: "thesis"
%%% End: 

\appendix

\chapter{源代码(模块代码主体部分)}
%定义源代码的排版样式
\lstset{tabsize=4,
  frame=none,
  stringstyle=\ttfamily,
  numbers=left,
  numberstyle=\small,
  extendedchars=false,columns=flexible,mathescape=true
  numbersep=-1em
}

%main.cpp
\lstinputlisting{src/meter.cpp}
%position.cpp
\lstinputlisting{src/position.cpp}
%unionfind.cpp
\lstinputlisting{src/unionfind.cpp}


%%% Local Variables: 
%%% mode: latex
%%% TeX-master: "thesis"
%%% End: 


\chapter{翻译(原文和译文)}


\end{document}

