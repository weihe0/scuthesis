
\chapter{机器视觉库OpenCV}

\section{概述}

本文的算法实现基于OpenCV,OpenCV降低了开发图像识别程序的难度,提高了计算效率。OpenCV的全称是open source computer vision library,是一个开源的计算机视觉库,用于图像处理、机器视觉、模式识别、机器学习、计算机建模和虚拟现实等领域。OpenCV由英特尔公司发起,旨在为CPU计算密集型任务提供软件支持。OpenCV具有如下特性:
\begin{asparaenum}[(1)]
\item 提供大量的算法。OpenCV提供人机交互、图像分割、运动追踪、三维建模、特征识别、目标识别、图像匹配、机器学习等领域的大量算法,涵盖图像处理、机器视觉和模式识别的方方面面。
\item 用C/C++开发。在早期的OpenCV版本中,绝大部分程序由C语言开发。最新的版本主要用C++语言开发。与用C语言开发的早期版本相比相比,用C++开发的新版本在兼容性、类型安全、类型推断、自动内存管理和并行计算优化等方面有显著优势。
\item   为多种程序设计语言提供接口。OpenCV原生接口是是C++语言。为了与早期版本兼容,也提供了C语言接口。为了便于Python和Java程序员开发机器视觉应用,OpenCV也提供了Python和Java语言的接口。
\item 支持多种平台。OpenCV支持Windows,Linux,Mac OS,iOS和Android平台。
\item 效率高,实时性强。OpenCV专注于计算效率和实时性,因此OpenCV提供的算法实现具有较快的速度。同时OpenCV针对多核CPU、CUDA和OpenCL等并行计算平台进行优化,进一步加快了OpenCV的计算速度。
\end{asparaenum}

\section{基本数据结构}

图像处理和识别需要进行大量的矩阵运算。OpenCV用Mat类表示矩阵。Mat类能表示任意大小、任意维度的矩阵,矩阵的元素可以为字节、整数、浮点数,而且Mat类表示的一维矩阵可以和点、向量、数字等数据结构进行相互转化。Mat类使用引用计数进行自动内存管理。创建矩阵时,Mat类


\section{图像滤波}

\section{几何变换}

\section{结构分析}

\section{特征算子}


%%% Local Variables: 
%%% mode: latex
%%% TeX-master: "thesis"
%%% End: 

