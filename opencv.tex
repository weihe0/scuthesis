
\chapter{机器视觉库OpenCV}

\section{概述}

本文的算法实现基于OpenCV,OpenCV降低了开发图像识别程序的难度,提高了计算效率。OpenCV的全称是open source computer vision library,是一个开源的计算机视觉库,用于图像处理、机器视觉、模式识别、机器学习、计算机建模和虚拟现实等领域。OpenCV由英特尔公司发起,旨在为CPU计算密集型任务提供软件支持。OpenCV具有如下特性:
\begin{asparaenum}[(1)]
\item 提供大量的算法。OpenCV提供人机交互、图像分割、运动追踪、三维建模、特征识别、目标识别、图像匹配、机器学习等领域的大量算法,涵盖图像处理、机器视觉和模式识别的方方面面。
\item 用C/C++开发。在早期的OpenCV版本中,绝大部分程序由C语言开发。最新的版本主要用C++语言开发。与用C语言开发的早期版本相比相比,用C++开发的新版本在兼容性、类型安全、类型推断、自动内存管理和并行计算优化等方面有显著优势。
\item   为多种程序设计语言提供接口。OpenCV原生接口是是C++语言。为了与早期版本兼容,也提供了C语言接口。为了便于Python和Java程序员开发机器视觉应用,OpenCV也提供了Python和Java语言的接口。
\item 支持多种平台。OpenCV支持Windows,Linux,Mac OS,iOS和Android平台。
\item 效率高,实时性强。OpenCV专注于计算效率和实时性,因此OpenCV提供的算法实现具有较快的速度。同时OpenCV针对多核CPU、CUDA和OpenCL等并行计算平台进行优化,进一步加快了OpenCV的计算速度。
\end{asparaenum}

\section{基本数据结构}

\subsection{向量数据结构}

在图像处理和机器视觉中,OpenCV用几种向量数据结构表示点、矩阵的行列、颜色值、特征向量等。表示点的数据结构有二维坐标类Point\_和三维坐标类Point\_。这两个类都支持坐标的加减运算$P_1\pm P_2$,坐标的缩放运算$kP$等。Vec类是通用的向量类,它表示任意长度、任意类型元素的向量。Scalar类是Vec类的特例,表示包含分量不多于四个的双精度浮点数向量。在OpenCV的库函数中,Scalar类一般用于指定颜色值,如在绘制线段的函数lines中,如果指定Scalar类为$(100,100,100)$,则线段颜色的RGB值均为100。

\subsection{矩阵数据结构}


图像处理和识别需要进行大量的矩阵运算。OpenCV用Mat类表示矩阵。Mat类能表示任意大小、任意维度的矩阵,矩阵的元素可以为字节、整数、浮点数。Mat类也可以表示多通道的矩阵,其每个元素为向量,用于记录该处所有通道的值。一维矩阵的Mat类可以和点、向量、数字等数据结构进行相互转化。由于Mat类表示的矩阵类型多种多样,一维向量、二维矩阵、灰度图像、彩色图像、点云、张量、直方图等数据均可用Mat类存储。

Mat类使用引用计数进行自动内存管理,使用户不必手动进行内存管理,提高了内存管理效率,避免了内存泄露和无效指针等内存管理错误。创建矩阵时,Mat类为矩阵数据分配内存,然后将引用计数置为1。矩阵占用大量内存,为了减少内存使用量,Mat类对象进行赋值运算$A=B$时,仅仅将$B$维数、大小等信息复制到$A$中,让$B$和$A$共享矩阵数据,然后将相应的引用计数加1。销毁Mat类对象时,先将其矩阵数据的引用计数减1,只有引用计数为0,即没有Mat类对象指向这块数据时,才释放数据。

Mat类支持许多矩阵运算,如矩阵加减$A\pm B$,矩阵与标量的乘法$kA$,矩阵与矩阵的乘法$A\times B$,转置$A^T$。如果两个Mat类对象都表示向量,还能计算向量的点积$A\cdot B$和叉积$A\times B$。除上述双目运算外,Mat类还支持几个单目运算,如矩阵的模、矩阵元素总和及均值、矩阵的迹和行列式等。

由于Mat类表示的矩阵多种多样,其构造函数也有多个。最简单的是Mat(),表示创建一个$3\time 3$矩阵。这个简单的构造函数广泛地用于滤波和形态学运算。另一个常见的构造函数是Mat(int rows, int cols, int type),表示创建一个行数为rows,列数为cols,元素类型为type的二维矩阵。一维矩阵的Mat类可以用构造函数Mat(Vec \& vec)从向量vec转化而来。高维矩阵的Mat类可以用构造函数Mat(int ndims, int sizes[], int type)创建,其中ndims,sizes[]和type分别表示维数、每个维度的行数和元素类型。

除了Mat类外,还有其他几个类可以表示矩阵。例如表示小尺寸矩阵的Matx类。Matx类在编译时就确定矩阵的大小和元素类型,适用于表示滤波模板和形态学的结构元等大小和元素值确定的模板。Matx类支持Mat类的大部分运算。如果某种运算在Matx类中没有实现,可以将Matx类转换成Mat类再计算。另一个表示矩阵的类是SparseMat。它内部只存储所有非零元素,因此特别适合表示大型高维稀疏矩阵。

\section{相关模块}

OpenCV实现了图像处理、机器视觉和模式识别的许多常见算法。由于实现的算法繁多,OpenCV按领域分成多个模块。目前OpenCV包括核心、图像处理、图形界面、视频分析、三维相机定标、特征识别、目标识别、机器学习、GPU加速等模块。本文使用了OpenCV的图像处理、图形界面和机器学习这三个模块。

OpenCV的图像处理模块涉及许多常见的图像处理方法,如滤波、几何变换、颜色空间变换、二值化、边缘检测和直方图变换等。在滤波子模块,OpenCV除了提供常见的滤波器外,还支持用户自定义的滤波器。自定义的滤波器可以合并其他处理步骤,如颜色空间转换和二值化等,从而提高程序性能。OpenCV提供了多种二值化算法,支持多个颜色空间之间的转换,包含了多种边缘检测算子和直方图变换技术,便于开发图像处理程序。

在图形界面模块,OpenCV支持创建简单窗口并在窗口显示图像。为了便于在图像处理程序中调整参数,OpenCV提供了创建滑块的函数,通过移动滑块控制参数。OpenCV的图形界面函数非常简单,调用几个函数就可以创建有简单的交互功能的图像处理程序,使用户不必了解图形界面的细节,将精力集中到实现图像处理算法中。

在机器学习模块,OpenCV实现了贝叶斯分类、k近邻算法、支持向量机、决策树和神经网络等多种机器学习算法。只需提供样本,OpenCV就能计算分类器的参数,对未知样本分类。

%%% Local Variables: 
%%% mode: latex
%%% TeX-master: "thesis"
%%% End: 

