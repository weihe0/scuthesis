\begin{center}
\vspace*{12pt}
{\bf\zihao{-2}The Algorithm and Implementation of \\[10pt]Ammeter Digit Recognition}\\[18pt]
\zihao{4}Computer Science and Technology\\[12pt]
{\bf Student} \quad Wei HE \quad {\bf Advisor} \quad Lei ZHANG\\[24pt]
\end{center}\par

{\zihao{5}[{\bf Abstract}]Image is an important form of information. Automatic image processing and recognition is an important field. In recent decades, as the storages volumn and computation speed of computers have been considerably improved, the obstacles that once prevented the application of image processing and computer vision no longer exist. Because of thia, Image processing and computer vision has been widely employed in plate number recognition, medical image processing, security and manufacturing. For instance, in manufacting, computer vision is often used to inspect the defectes of products. In door security, image processing is used to recognise the identity of guests. In libraries and archieves, printed essays are scanned and their texts are automatically recognised.

There are many semi-mechanical ammeters in a power station. Due to the face that these ammters are scattered and located in high voltage environment, it is arduous and adventurous for workers to read the meters. These semi-machanical ammeters cannot be replaced by electronic ammters in short period, as they are considerably large and installed in cabinet tightly. Because of this, in order to reduce the workload and achieve real-time monitoring, it's essential to recognise the digits in the images of them.

The cameras in front of them capture color image. The digits in the images takes only a small part, which is surrounded by several areas, causing interference in recognition. Therefore, it is a key point to seperate digits and other parts. Considering these factes, the thesis proposed that digit recognition be divided into four stages consequently: preprocessing, frame positioning, digit seperation and digit recognition. During preprocessing stage, the color image is converted to gray-scale image at first. After that, we perform gaussion blur and histogram equalization. During the stage of frame positioning, consider the fact that black and white areas contribute to the most part of the image, we can use Otsu algorithm to acquire binary image. Then we use connected components labeling algorithms to find out the frame. Since the frame is slanted, we invoke Hough transform to determine the slope and calibrate the frame. In the segmentation stage, the thesis proposed a threshold based segmenting algorithm. After that, we perform projections in both horizontal and vertical axis to locate the images. In the last stage, as the digits in ammters are consistent, the thesis proposed an template matching algorithm to recognise the digits. The experiments have proved the effect of the algorithms mentioned above.
}

\vspace{10pt}

[{\bf Key Words}]{Ammeter, Image Segmentation, Charactor Recognition}

%%% Local Variables: 
%%% mode: latex
%%% TeX-master: "thesis"
%%% End: 

