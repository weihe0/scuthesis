\begin{center}
\vspace*{12pt}
{\bf\zihao{-2}Automatic Ammeter Digit Recognition}\\[18pt]
\zihao{4}Computer Science and Technology\\[12pt]
{\bf Student} \quad Wei HE \quad {\bf Advisor} \quad Lei ZHANG\\[24pt]
\end{center}\par

{\zihao{5}[{\bf Abstract}] Automatic image processing and recognition is an important field. In recent decades, as the storages volumn and computation speed of computers have been considerably improved, the obstacles that once prevented the application of image processing and computer vision no longer exist. Because of this, Image processing and computer vision has been widely employed in plate number recognition, medical image processing, security and manufacturing. For instance, in manufacturing, computer vision is often used to inspect the defects of products. In , image processing is used to recognise the identity of guests. In libraries and archieves, printed essays are scanned and their texts are automatically recognised.

There are many semi-mechanical ammeters in a power station. Due to the fact that these ammeters are scattered and located in high voltage environment, it is arduous and hazardous for workers to read the meters. These semi-machanical ammeters cannot be replaced by electronic ammters in short period, as they are considerably large and installed in cabinet tightly. Because of this, in order to reduce the workload and achieve real-time monitoring, it's essential to recognise the digits in the images of them.

The cameras in front of them capture color image. The digits in the images takes only a small part, which is surrounded by several areas, causing interference in recognition. Therefore, it is a key point to separate digits and other parts. Considering these facts, the thesis propose  be divided into four stages consequently: preprocessing, frame positioning, digit seperation and digit recognition. During preprocessing stage, the color image is converted to gray-scale image at first. After that, we perform Gaussian blur and histogram equalization. During the stage of frame positioning, we can use Otsu algorithm to acquire binary image. Then we use connected components labeling algorithms to find out the frame. Since the frame is slanted, we apply Hough transform to determine the slope and calibrate the frame. In the segmentation stage, the thesis proposes a threshold based segmenting algorithm. After that, projections on both horizontal and vertical axis are performed to locate the digits in images. In the last stage, as the digits in ammeters are consistent, the thesis propose template matching algorithm to recognise the them. The experiments have proved the effect of the algorithms.
}

\vspace{10pt}

[{\bf Key Words}]{Ammeter, Image Segmentation, Charactor Recognition}

%%% Local Variables: 
%%% mode: latex
%%% TeX-master: "thesis"
%%% End: 

