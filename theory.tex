
\chapter{图像处理基本理论}

电流表识别的各个环节都设计图像处理的基本理论。

\section{图像增强}

实际图像不是完美的。由于噪声和光照等原因,图像的质量达不到自动识别的要求,所以需要进行图像增强。图像增强可以突出待识别的目标的特征,抑制噪声和目标外的部分,从而使增强的图像更适合计算机识别。常见的图像增强方法有\emph{灰度变换}和\emph{图像平滑}。

\subsection{灰度变换}

灰度变换就是将图像每个像素的灰度变换成另一种灰度。灰度变换后每个像素的灰度仅仅由变换前的灰度决定。如果输入图像为$F(i,j)$,输出图像为$G(i,j)$,则灰度变换可以表示为:
\begin{equation}
  G(i,j)=T(F(i,j))
\end{equation}

变换函数$T(x)$可以由用户指定,也可以根据直方图确定。\emph{直方图均衡}就是一种根据直方图确定变换函数的方法。直方图均衡主要用于增加图像的全局对比度,改善光照条件不佳的图像。例如对背景或前景过亮或过暗的图像进行光照补偿,增强曝光过度或曝光不足的图像的细节等。直方图均衡化使输出图像包括所有可能的灰度级,并在每个灰度级上有大致相等的像素个数。要达到上述要求,首先求出直方图。设$n$表示图像的像素数,$n_i$表示灰度$i$出现的次数。灰度$i$有$0,1,\cdots,L-1$共$L$个灰度级。则直方图均衡化的灰度变换函数$T(x)$可以用如下方式求出。直方图$p(i)$表示灰度$i$出现的概率,用公式表示为:
\begin{equation}
  \label{eq:hist}
  p(i)=\frac{n_i}{n},i=0,1,\cdots,L-1
\end{equation}
再从直方图求出累计概率函数$c(i)$,定义如下:
begin{equation}
  \label{eq:acc}
  c(i)=\sum_{j=0}^i p(j)
\end{equation}
最后将$c(i)$映射到$L$个灰度级:
\begin{equation}
  \label{eq:map}
  m(c(i))=\left[\frac{c(i)-\min(c(i))}{\max(c(i))-\min(c(i))}\cdot L\right]
\end{equation}
此时得到灰度变换函数为$T(x)=m(c(x))$。

\subsection{图像平滑}

原图像含有较多的噪声。噪声一般表现为孤立的点和线。用图像平滑的方法,使图像模糊化。去除孤立的点线。图像平滑对像素的邻域进行运算。这种运算可以用模板表示。\emph{模板}是表示领域内像素位置对应的值的方法。


将每个像素的灰度用其邻域的算术平均值代替,可以减少图像的噪声,实现对图像的平滑处理。这种方法称为\emph{盒形滤波}。设$F(i,j)$表示输入图像,$G(i,j)$表示平滑的图像,$S$表示邻域的像素坐标集合,$M$表示邻域的像素数目,则盒形滤波可以用公式表示为:



\section{边缘检测}

\subsection{一维信号边缘检测}

\subsection{二维图像边缘检测算子}

\subsubsection{简单算子}

\subsubsection{Canny算子}


\section{图像分割}

\subsection{阈值分割}

\subsection{轮廓分割}

\section{二值图像分析}

\subsection{连通成分标记}

\subsection{形态学运算}
%%% Local Variables: 
%%% mode: latex
%%% TeX-master: "thesis"
%%% End: 
