
\chapter*{致\qquad谢}\addcontentsline{toc}{chapter}{致谢}

首先感谢Linda Shapiro和George Stockman。他们合著的《计算机视觉》深入浅出地讲述了机器视觉的基本理论和技术。我就是从这本书学到图像处理和计算机视觉的基础知识。另外,Wesley Snyder和Hairong Qi合著的《机器视觉教程》内容浅显,让我在短时间内就掌握了机器视觉的基本算法的要领。刘直芳、王运琼和朱敏老师合编的《数字图像处理与分析》包含了丰富的公式。本文的部分公式就来自这本书。张宏林编著的《Visual C++数字图像模式识别技术与工程实践》和苏彦华编著的《Visual C++数字图像识别技术典型案例》包含了大量的源代码。通过阅读源代码,我实现了自己的毕业设计。在此对他们提出感谢。

毕业论文和设计里大量使用了OpenCV的函数。感谢OpenCV的开发者Vadim Pisarevsky等人。OpenCV提供了许多常见的图像处理和机器视觉算法,大大降低了我开发图像识别程序的难度。为学习OpenCV,我参考了《Mastering OpenCV with Practical Computer Vision Projects》和《OpenCV Cookbook》两本书。为此,向下面这几位作者致谢:Daniel Lélis Baggio,Shervin Emami,David Millán Escrivá,Khvedchenia Ievgen,Naureen Mahmood,Jason Saragih,Roy Shilkrot,Robert Laganière。

本文是用\XeLaTeX~排版的。\XeLaTeX~由基础排版引擎\XeTeX~和高层格式包\LaTeX~组成,而\XeTeX~排版引擎源于\TeX~。所以这里感谢\XeTeX~的开发者Jonathan Kew,\LaTeX~的开发者Leslie Lamport和Frank Mittelbach,以及\TeX~的开发者Donald Knuth。\XeLaTeX~本身的功能有限,仅仅用它是无法完成论文排版的。因此我使用了大量的\LaTeX的扩展包,其中最重要的是\LaTeX中文排版扩展包ctex,所以这里特别感谢开发和维护ctex的中文\TeX~学会。在使用\LaTeX~的过程中,《\LaTeXe~完全学习手册》给了我极大的帮助。这本书详细地解释了常见的\LaTeX~命令。每当我用\LaTeX排版遇到问题时,在这本书里总能找到解决方案,所以这里特别感谢这本书的作者胡伟。本文的插图是用与\TeX~排版系统配套的\MP制作的。感谢\MP的作者John D. Hobby和Taco Hoekwater。没有他们,用\XeLaTeX排版出精美的毕业论文,是难以想象的。

感谢指导毕业论文的张蕾老师和郭泉师兄。在和郭泉师兄的交流中,我逐渐明白了图像识别系统的原理和常见问题。另外感谢卢晓春老师,她提出了许多有关图像识别的宝贵建议,使我找到了合适的算法。最后感谢父母对我的关怀和支持。
%%% Local Variables: 
%%% mode: latex
%%% TeX-master: "thesis"
%%% End: 
