
\chapter{总结与展望}

\section{总结}


由于电流表图像中,与读数无关的部分太多,图像中也含有较多噪声,将读数识别分成图像预处理、数字边框提取、数字分割和数字识别这四个部分。

在预处理部分,为增强图像质量,减少噪声,针对电流表图像的特点,使用了几种常见的图像处理方法。摄像头拍摄的图像是彩色图像,处理比较困难,因此先转换成灰度图像。由于电流表图像采集的时间不同,图像的光照条件也不相同,为了减少光照条件变化带来的影响,使各个时间的图像的灰度大致相同,我们采用直方图均衡化处理图像。然后观察图像噪声的特点,发现主要是孤立的点线噪声。根据噪声的这个特点,本文提出进行高斯滤波以见效噪声。

在数字边框提取部分,首先根据仪表可以分成大片的白色和黑色区域的情况,采用Otsu阈值法,找出黑色区域。然后找出这些像素形成的若干个连通成分。为了减少计算开销,本文逐行扫描算法。然后用连通成分的长宽比、像素密度、像素数目和面积等指标筛选连通成分,找出数字边框对应的连通成分。由于数字边框由一定的倾斜角度,先做水平方向的边缘检测,再利用改进的哈夫变换找出上下两条横线的倾斜角度,然后根据倾斜角度利用仿射变换将数字边框图像旋转回原位置。

在数字分割部分,根据背景不均匀的特点,本文分使用了基于动态阈值的数字分割算法。首先用动态阈值法粗略地找出数字对应的像素,再用形态学的方法填补数字的空洞和缝隙,消除错分为数字像素的背景像素。然后做水平和竖直方向上的投影,通过寻找波峰和波谷,找出数字的位置。

在数字识别部分,本文针对仪表数字大小形状一致和数字分割中易出现断裂和缺口的情况,采用模板匹配法识别数字。在进行模板匹配哦前,需要对待识别图像进行归一化,将数字的尺寸、大小和笔画粗细统一。其中在笔画粗细归一化使用Hilditch算法。最后在匹配模板时,使用Hausdorff距离度量待识别数字和模板的差距,取差距最小的模板对应的数字最为识别结果。

\section{展望}

本文提出了电流表读数识别的算法。在实际应用中,电流表图像的噪声大、干扰多,该算法的准确性和鲁帮性还有待提高。要使上述算法能真正用于电站的电流表自动识别,仍需要做许多细致而深入的工作。

在预处理阶段,部分电流表图像的噪声和干扰比较多,需要使用更有效的图像增强算法。既保证消除噪声和干扰,又不能使待识别的数字模糊。在数字边框定位阶段,Otsu算法对大部分图像都有较好的效果,但对少部分电流表保护玻璃的反光强烈的图像,Otsu算法失效。因此,对于反光强烈的图像,需要使用更有效的算法定位边框。在数字分割阶段,使用局部阈值法得到的二值化图像有许多干扰点线。只有将这些干扰点线几乎全部去除,投影法定位的数字才准确。最后,在数字识别阶段,如何有效地计算数字字符图像和模板的差距,是一个值得考虑的问题。

电流表的自动识别算法可以推广到其他类型仪表的自动识别。然而,各种仪表数字的大小、位置和形状各不相同。如何推广电流表的自动识别算法,找出一般的仪表自动识别的通用规律,也是一个值得探索的问题。
%%% Local Variables: 
%%% mode: latex
%%% TeX-master: "thesis"
%%% End: 
