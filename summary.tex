
\chapter*{总\qquad结}\addcontentsline{toc}{chapter}{总结}

随着图像处理技术的迅速发展,越來越多的识别任务用计算机自动完成。计算机识别具有精度高、速度快的优点,从而减少人力,实现自动化抄表。

电流表图像的字符识别可以分成图像预处理、数字边框提取、数字分割和数字识别这四个部分。

在预处理部分,为增强图像质量,减少噪声,针对电流表图像的特点,介绍并比较了几种常见的图像处理方法。摄像头拍摄的图像是彩色图像,处理比较麻烦,因此先转换成灰度图像。由于电流表图像采集的时间不同,图像的光照条件也不相同,为了减少光照条件变化带来的影响,使各个时间的图像的灰度大致相同,我们采用直方图均衡化处理图像。然后观察图像噪声的特点,发现主要是孤立的点线噪声。根据噪声的这个特点,本文提出先进行高斯滤波,再进行中值滤波平滑图像。

在数字边框提取部分,首先根据数字边框的背景为黑色的情况,采用全局阈值法,设定一个较低的阈值,找出低于阈值的像素。然后找出这些像素形成的若干个连通成分,介绍并比较了两种连通成分标记算法。为了减少计算开销,本文逐行扫描算法。然后用连通成分的长宽比、像素密度、像素数目和面积等指标筛选连通成分,找出数字边框对应的连通成分。由于数字边框由一定的倾斜角度,先做水平方向的边缘检测,再利用改进的哈夫变换找出上下两条横线的倾斜角度,然后根据倾斜角度利用仿射变换将数字边框图像旋转回原位置,最后用双线性插值法填补旋转后图像的空洞。

在数字分割部分,根据背景不均匀的特点,本文分别介绍了基于边缘和基于动态阈值的数字分割算法。在基于边缘的数字分割算法中,首先介绍并比较了几种边缘检测算子,选择Canny算子做边缘检测,然后提取轮廓,然后去掉高度和宽度低于设定值的虚假轮廓,从而找出数字的轮廓,最后填充轮廓以完成分割。在基于动态阈值的数字分割算法中,首先用动态阈值法粗略地找出数字,再用形态学的方法填补数字的空洞和缝隙,然后做水平和竖直方向上的投影,通过寻找波峰和波谷,找出数字的位置。

在数字识别部分,本文分别介绍并比较了基于模板的和基于特征的匹配算法。在基于木板匹配的方法中,首先用仿射变换将数字高度缩放到和模板相同,然后定义模板匹配准则,将数字和所有模板比较,找出最相似的模板对应的数字。在基于特征的匹配算法中,首先提取特征向量,然后用决策树对特征向量分类,找出特征向量对应的数字。
%%% Local Variables: 
%%% mode: latex
%%% TeX-master: "thesis"
%%% End: 
