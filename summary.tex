
\chapter{总结}

由于电流表图像中,与读数无关的部分太多,图像中也含有较多噪声,将读数识别分成图像预处理、数字边框提取、数字分割和数字识别这四个部分。

在预处理部分,为增强图像质量,减少噪声,针对电流表图像的特点,使用了几种常见的图像处理方法。摄像头拍摄的图像是彩色图像,处理比较困难,因此先转换成灰度图像。由于电流表图像采集的时间不同,图像的光照条件也不相同,为了减少光照条件变化带来的影响,使各个时间的图像的灰度大致相同,我们采用直方图均衡化处理图像。然后观察图像噪声的特点,发现主要是孤立的点线噪声。根据噪声的这个特点,本文提出进行高斯滤波以见效噪声。

在数字边框提取部分,首先根据仪表可以分成大片的白色和黑色区域的情况,采用Otsu阈值法,找出黑色区域。然后找出这些像素形成的若干个连通成分。为了减少计算开销,本文逐行扫描算法。然后用连通成分的长宽比、像素密度、像素数目和面积等指标筛选连通成分,找出数字边框对应的连通成分。由于数字边框由一定的倾斜角度,先做水平方向的边缘检测,再利用改进的哈夫变换找出上下两条横线的倾斜角度,然后根据倾斜角度利用仿射变换将数字边框图像旋转回原位置。

在数字分割部分,根据背景不均匀的特点,本文分使用了基于动态阈值的数字分割算法。首先用动态阈值法粗略地找出数字对应的像素,再用形态学的方法填补数字的空洞和缝隙,消除错分为数字像素的背景像素。然后做水平和竖直方向上的投影,通过寻找波峰和波谷,找出数字的位置。

在数字识别部分,本文针对仪表数字大小形状一致和数字分割中易出现断裂和缺口的情况,采用模板匹配法识别数字。在进行模板匹配哦前,需要对待识别图像进行归一化,将数字的尺寸、大小和笔画粗细统一。其中在笔画粗细归一化使用Hilditch算法。最后在匹配模板时,使用Hausdorff距离度量待识别数字和模板的差距,取差距最小的模板对应的数字最为识别结果。

%%% Local Variables: 
%%% mode: latex
%%% TeX-master: "thesis"
%%% End: 
