
\chapter{源代码(模块代码主体部分)}
%定义源代码的排版样式
\lstset{
  tabsize=4,
  frame=none,
  stringstyle=\ttfamily,
  numbers=left,
  numberstyle=\small,
  extendedchars=false,columns=flexible,mathescape=true
  numbersep=-1em
}

meter.cpp节选
\lstinputlisting{src/meter.cpp}
position.cpp节选
\begin{lstlisting}
bool findWheel(Mat &src, Mat &dst)
{
    Mat bin, binInv;
    threshold(src, bin, 60.0, 255.0, THRESH_BINARY | THRESH_OTSU);
    imwrite("otsu.png", bin);
    threshold(src, binInv, 60.0, 255.0, THRESH_BINARY_INV | THRESH_OTSU);
    vector<vector<Point> > contours;
    findContours(binInv, contours, CV_RETR_EXTERNAL, CV_CHAIN_APPROX_NONE);
    int idx = selectContour(contours);

    if (idx < 0)
    {
	return false;
    }

    Rect rect = boundingRect(contours[idx]);
    Mat mask(bin.size(), bin.type(), Scalar(0));
    drawContours(mask, contours, idx, Scalar(255), CV_FILLED);
    Mat left;
    src.copyTo(left, mask);
    Mat rotated = src(rect).clone();

    double slant = slantAngle(contours[idx]);
    Point center;
    center.x = rotated.cols / 2;
    center.y = rotated.rows / 2;
    Mat rot = getRotationMatrix2D(center, slant, 1.0);

    warpAffine(rotated, dst, rot, rotated.size());    
    imwrite("frame.png", rotated);
    imwrite("rotate.png", dst);

    return true;
}
\end{lstlisting}
segment.cpp节选
\begin{lstlisting}
void segment(Mat &src, vector<Mat> &digitVec)
{
    Mat bin;
    adaptiveThreshold(src, bin, 255.0, ADAPTIVE_THRESH_GAUSSIAN_C,
		      THRESH_BINARY, 5, -1.5);
    medianBlur(bin, bin, 3);
    morphologyEx(bin, bin, MORPH_CLOSE, Mat());
    imshow("bin", bin);

    MatND px;
    Mat pxImg;
    projectHor(bin, px, 1);
    drawProject(px, pxImg);
    imwrite("projectx.png", pxImg);
    // imshow("horizontal", pxImg);

    trimPro(px, px);
    histerThresh(px, px, 15, 3);
    drawProject(px, pxImg);
    //imshow("thresh_project", pxImg);

    vector<Vec2i> horPeaks;
    findPeaks(px, horPeaks, 5, 35, 2);

    const char *names[10] = {"1.png", "2.png", "3.png", "4.png", "5.png",
     			    "6.png", "7.png", "8.png", "9.png", "10.png"};
     for (int i = 0; i < horPeaks.size(); i++)
     {
       if (horPeaks[i][0] < 10 || horPeaks[i][1] < 15)
       {
	   continue;
       }
      	Mat candDigit = bin(Range(0, bin.rows),
      			Range(horPeaks[i][0], horPeaks[i][1])).clone();
	cout << horPeaks[i][0] << "\t" << horPeaks[i][1] << endl;
 	MatND py;
     	vector<Vec2i> verPeaks;
     	projectVer(candDigit, py, 1);
     	findPeaks(py, verPeaks, 15, 50, 1);

	if (verPeaks.size() > 0)
	{
	    Mat digit = candDigit(Range(verPeaks[0][0], verPeaks[0][1]),
				  Range(0, candDigit.cols));
	    Rect rect(horPeaks[i][0], verPeaks[0][0], 
		      horPeaks[i][1] - horPeaks[i][0],
		      verPeaks[0][1] - verPeaks[0][0]);
	    rectangle(bin, rect.tl(), rect.br(), Scalar(128), 2);
	    digitVec.push_back(digit);
	}
     }
     imshow("bin", bin);
}
\end{lstlisting}

%%% Local Variables: 
%%% mode: latex
%%% TeX-master: "thesis"
%%% End: 
