
\chapter{绪论}\pagenumbering{arabic}

\section{课题的背景}

各种各样的仪表用于测量系统,其中大多数仪表是机械式和半机械式的。这些仪表不能以数字化的形式存储读数,也没有数据接口传输读数。这些仪表的读数需要人工抄录。在有些场合下,人工抄录是可行的。但在另一些场合下,人工抄录不能满足要求。在需要实时监控的场合,用人工抄录的办法不能及时使响应读数的异常变化,也很难做到全天24小时监控。在化工、钢铁、冶金和核辐射等危险场合,人工读数威胁到工作人员的安全,是不现实的做法。

随着图像处理、模式识别和机器视觉等技术的发展,许多图像识别任务都可以使用计算机自动完成。例如在制造业中,常常利用机器视觉方法自动地检测产品的质量,从而极大地提高生产的自动化程度;在海关、自动柜员机、门禁系统中,通过扫描虹膜或人眼,计算机系统能识别身份;在银行业,机器视觉方法用于自动地识别印章真伪;档案管理系统也可以用机器视觉技术将纸质文档数字化,以便于存储和检索。仪表读数识别也可以利用机器视觉技术完成。通过自动识别仪表读数,可以方便地进行集中管理,及时地对异常情况进行处理。



\section{课题的意义}

本文要识别的电流表来自于发电厂。这些电流表都是半机械式的。由于电流表在高压环境中,人工抄表有较大危险。另一方面,这些电流表体积庞大,且安装在整套的仪表柜中,很难取出,这些半机械式的电流表暂时无法替换成电子式电流表。为了实现实时监控,可以在电流表前安装摄像头,实时采集电流表图像,然后用计算机系统识别。这种方式能减少工作量,实现无人值守的电流监控。随着电子技术和计算机技术的发展,各种图像采集系统和机器识别系统的性能越來越高,价格也具有很强的竞争力。因此,用仪表读数的自动识别代替人工抄表,能降低人力成本,也可以减少高电压的危险环境对工作人员的伤害。

%基于机器视觉的远程仪表自动抄表系统的出现将极大地提高抄表的速度和准确率性,进而提高工作效率,降低企业经营成本,同时对于这些特定的危险环境,可以减少对人的伤害。通过齿轮来显示表盘数字的仪表在读取的过程中有着相似的显示规律,因此基于机器视觉的仪表数字识别研究在某些特定的实际生产生活中有着非常广阔的应用前景。

电流表的仪表读数识别技术也可以推广到其他仪表的读数识别。部分仪表由于其工作原理和测量环境的特殊性,不宜安装电子线路,如油位计、流量计和气表等,为了实现自动抄表,采集这些仪表的图像进行读数识别是一个可行的选择。在核电站和化工厂等危险场所,如果进行人工抄表,工作人员可能受到较大的伤害。利用机器视觉技术实现远程抄表,可以减少对工作人员的危害。对仪表实现自动读数后,与计算机系统连接,实现管控自动化,进而提高工作效率,降低管理成本。

\section{相关工作}

仪表读数识别涉及数字图像处理技术。随着计算机的计算速度和存储能力的迅速发展,图像处理所需的极高的计算和存储要求得到满足。限制图像处理技术应用的因素大多被消除。 数字图像处理技术得到广泛应用。

仪表读数识别和光学字符识别有着很紧密的联系。早在1929年,德国就出现了第一份有关光学字符识别的专利。到五十年代计算机出现时,IBM公司便开始光学字符识别的研究,当时用于扫描美国政府浩如烟海的纸质档案,以文本的形式储存在计算机中。从八十年代开始,CCD和CMOS等图像采集器件和大规模集成电路的出现,使数字图像的采集、存储和处理技术跃上了一个新的台阶,光学字符识别技术也随之得到极大的发展。光学字符识别技术用于档案扫描、信件分拣、车牌识别等方面。特别值得一提的是由惠普研究院研制的tesseract-ocr软件。它能读入多种格式的图像,并能识别超过60种语言的文本图像。在光学字符识别精度的竞赛中,它也名列前茅。

我国在数字识别方面的研究从八十年代开始,起步比国外晚。几十年来,科研人员付出了艰辛的劳动,取得了丰硕的成果。光学字符识别技术取得了长足的进步。如今,对于常见的光学字符应用,已经有了成熟的解决方案。在交通领域,光学字符技术广泛地用于车牌识别。在印刷体字符识别领域,由清华大学研发的TH-OCR获得了业界广泛的认可。目前,光学字符识别理论非常成熟,对于较清晰的字符,识别率已接近100\%,这些为仪表读数的计算机识别奠定了坚实的基础。

%%% Local Variables: 
%%% mode: latex
%%% TeX-master: "thesis"
%%% End: 
